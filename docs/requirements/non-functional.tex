\section{Non-Functional Requirements}

\subsection{Accessibility}
Help text will be provided in English. At all times, a keyboard shortcut will
be available to the user that provides useful information about game flow and
controls.

\subsection{Compatibility}
Backwards compatibility between major version numbers (as specified by the
Semantic Versioning 2.0.0 guidelines) is not guaranteed. Incompatible clients
will be filtered out of the discovered games view.

\subsection{Performance}
Latency between user interactions should be no longer than 500ms. In the
case of network-based functionality, a 5000ms time limit shall be placed on
network queries. In instances where a network task takes longer than 5000ms to
complete, the user interaction will be canceled and the user who began the
query will be notified.

\subsection{Reliability}
The application should never enter into an error state where it crashes. All
errors ought to be logged or handled gracefully in such a way that the user
experiencing the error is able to understand the cause of the problem or report
error information to an entity capable of fixing or reproducing the bug.

\subsection{Safety}
The application ought never exceed its privileges or modify files on a client
machine that would otherwise impact performance or usability of said machine.

\subsection{Scalability}
The application is designed to scale up to four (4) clients.

\subsection{Testability}
The application is to be covered 100\% by unit tests, as reported by
\texttt{kcov}\cite{kcov} version \texttt{v33}.

\subsection{Usability}
Any user familiar with the game of checkers ought to be able to understand the
program well enough within 10 minutes of its opening that they can competently
play with another user.

\subsection{Versioning}
The application is required to use Rust version
\texttt{1.32.0}\cite{rust1.32.0} as its minimum compatible compiler version.
\\\\
\noindent The application itself is to use the Semantic Versioning 2.0.0\cite{semver} guideline, as
follows, for release versioning:
\begin{quote}
Given a version number \texttt{MAJOR.MINOR.PATCH}, increment the:
  \begin{enumerate}
    \item \texttt{MAJOR} version when you make incompatible API changes,
    \item \texttt{MINOR} version when you add functionality in a backwards-compatible manner, and
    \item \texttt{PATCH} version when you make backwards-compatible bug fixes.
  \end{enumerate}
\end{quote}
